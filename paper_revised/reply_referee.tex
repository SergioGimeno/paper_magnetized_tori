%
% version 27.10.16
%
%%%%%%%%%%%%%%%%%%%%%%%%%%%%%%%%%%%%%%%%%%%%%%%%%%%%%%%%%%%%%%%%%%%%%%
%%%%        reply
%%%%%%%%%%%%%%%%%%%%%%%%%%%%%%%%%%%%%%%%%%%%%%%%%%%%%%%%%%%%%%%%%%%%%%
  
\documentclass{article}
\usepackage{graphicx}
\usepackage{epsfig}
\usepackage{amsfonts}
\textheight 22.cm
\textwidth 16.5 cm
\oddsidemargin 0.5cm
\evensidemargin 0.5cm
\topmargin=-1.cm
\hoffset -0.5cm
 \usepackage{amssymb}

\tolerance=10000
\pagenumbering{arabic}
\textheight 22.cm
\textwidth 16.5 cm
\oddsidemargin 0.5cm\evensidemargin 0.5cm
\topmargin=-1.cm
\hoffset -0.5cm
\date{\today} 
 
\newcommand{\bnabla}{\mbox{\boldmath $\nabla$}}
%\newcommand{\bomega}{\mbox{\boldmath $\omega$}}
\newcommand{\bOmega}{\mbox{\boldmath $\Omega$}}

\newcommand{\insertplot}[5]{\begin{figure}
 \hfill\hbox to 0.05in{\vbox to #5in{\vfill
 \inputplot{#1}{#4}{#5}}\hfill}
 \hfill\vspace{-.1in}
 \caption{#2}\label{#3}
 \end{figure}}
 \newcommand{\inputplot}[3]{\special{ps:: gsave #2 -36 mul 0 rmoveto
   currentpoint translate #2 7.0 div #3 5.0 div scale}
 \special{ps: plotfile #1}\special{ps:: grestore}}
\newcounter{fig}   \newcommand{\lbfig}[1]{\refstepcounter{fig}
\label{#1} }
 
\newcommand{\vphi}{\varphi}
\newcommand{\DS}{\displaystyle}
\newcommand{\pih}{\frac{\pi}{2}}
\newcommand{\sqdetg}{\sqrt{-g}}
\newcommand{\sqdetgi}{\frac{1}{\sqrt{-g}}}
\newcommand{\edil}{e^{2\kappa \phi}}
\newcommand{\bA}{\bar{A}}
\newcommand{\bF}{\bar{F}}
\newcommand{\bD}{\bar{D}}

\renewcommand{\a}{\alpha}
\renewcommand{\b}{\beta}
\renewcommand{\c}{\gamma}
\renewcommand{\d}{\delta}  
\newcommand{\g}{\nu}
\renewcommand{\l}{\lambda}
\renewcommand{\t}{\theta}
\newcommand{\rot}{\mbox{rot}}
\renewcommand{\div}{\mbox{div}}
\newcommand{\cosec}{\mbox{cosec}}

 

\newcommand{\ta}{\theta}
\newcommand{\Si}{\Sigma}
\newcommand{\vf}{\varphi}
\newcommand{\dd}{\mbox{d}}
\newcommand{\tr}{\mbox{tr}}
\newcommand{\la}{\lambda}
\newcommand{\ka}{\kappa}
\newcommand{\f}{\phi}
\newcommand{\al}{\alpha}
\newcommand{\ga}{\gamma}
\newcommand{\de}{\delta}
\newcommand{\si}{\sigma}
\newcommand{\bomega}{\mbox{\boldmath $\omega$}}
\newcommand{\bsi}{\mbox{\boldmath $\sigma$}}
\newcommand{\bchi}{\mbox{\boldmath $\chi$}}
\newcommand{\bal}{\mbox{\boldmath $\alpha$}}
\newcommand{\bpsi}{\mbox{\boldmath $\psi$}}
\newcommand{\brho}{\mbox{\boldmath $\varrho$}}
\newcommand{\beps}{\mbox{\boldmath $\varepsilon$}}
\newcommand{\bxi}{\mbox{\boldmath $\xi$}}
\newcommand{\bbeta}{\mbox{\boldmath $\beta$}}
\newcommand{\ee}{\end{equation}}
\newcommand{\eea}{\end{eqnarray}}
\newcommand{\be}{\begin{equation}}
\newcommand{\bea}{\begin{eqnarray}}

\newcommand{\pont}{{\,^\ast\!}R\,R}

\newcommand{\ii}{\mbox{i}}
\newcommand{\e}{\mbox{e}}
\newcommand{\pa}{\partial}
\newcommand{\Om}{\Omega}
\newcommand{\vep}{\varepsilon}
%\newcommand{\bfR}{{\bf R}}
\newcommand{\bfph}{{\bf \phi}}
\newcommand{\lm}{\lambda}
\def\theequation{\arabic{equation}}
%\topmargin= -02cm\textheight= 23.cm\textwidth= 16.cm
%\oddsidemargin=-01cm\evensidemargin=-01cm \font\sqi=cmssq8
\renewcommand{\thefootnote}{\fnsymbol{footnote}}
\newcommand{\re}[1]{(\ref{#1})}
\newcommand{\R}{{\rm I \hspace{-0.52ex} R}}
\newcommand{\N}{{\sf N\hspace*{-1.0ex}\rule{0.15ex}%
{1.3ex}\hspace*{1.0ex}}}
\newcommand{\Q}{{\sf Q\hspace*{-1.1ex}\rule{0.15ex}%
{1.5ex}\hspace*{1.1ex}}}
\newcommand{\C}{{\sf C\hspace*{-0.9ex}\rule{0.15ex}%
{1.3ex}\hspace*{0.9ex}}}
\newcommand{\eins}{1\hspace{-0.56ex}{\rm I}}
\renewcommand{\thefootnote}{\arabic{footnote}}

\def\theequation{\thesection.\arabic{equation}}



\begin{document}

\title{ 
{\bf Reply to the  Referee report:} \\ {\large AA/2017/30935} \\  {\large  \textit{Magnetised Polish doughnuts revisited}\\ by Sergio Gimeno-Soler and Jos\'e A. Font.}
} 

\maketitle 

Dear A\&A Editor,

\bigskip

Thank you for sending us the report of the referee, whom we thank for
carefully reading our manuscript and providing very useful feedback. We 
have made several changes addressing the issues raised in the report, 
which are listed below together with our response. The changes made in 
the manuscript are highlighted in boldface, to facilitate identification. 
We hope that with these amendments the paper will be considered 
acceptable for publication on \textit{Astronomy and Astrophysics}.

\bigskip



Best wishes,

\bigskip



The authors

\bigskip

\section*{Referee report -- AA/2017/30935}

We are grateful to the referee for the interesting comments provided in his/her report.
We have followed the referee's suggestions and have incorporated most of his/her comments 
in our manuscript. Below, we ({\bf A}) address the various issues raised by the referee ({\bf R}).

\bigskip

{\bf R:} The paper presents a new procedure to construct sequences of equilibrium
tori around Kerr black holes that are endowed with a toroidal magnetic
field. These configurations are not expected to be produced under generic
conditions but the construction of such equilibria is interesting per se
and has a rather long history. Furthermore, as discussed by the authors,
the availability of such equilibria can be employed in test simulations
of general-relativistic codes in ideal MHD.The paper is nicely written with a clear discussion of the procedure
followed and of the properties of the tori that have been found. The
paper is essentially ready for publication but there are a few aspects
that could be improved and that I ask the authors to consider.

\bigskip

{\bf R:} - Title: I frankly find it unnecessarily colloquial. I do not insist but
I think a more pragmatic title such as ``A new method for the
construction of magnetized tori around black holes" is scientifically
more informative.

\bigskip

{\bf A:} We agree with the referee about the title being colloquial. Our choice was
motivated by the title of the paper of Qian et al (2009), ``The Polish 
doughnuts revisited …", of which our paper might be considered an extension 
(by incorporating magnetic fields). We would prefer to keep the original 
title, if the referee concurred. If that were not the case, we would not
object to having it changed to his/her suggestion. (``A new method for the
construction of magnetized tori around black holes".)

\bigskip

{\bf R:} - Sec. 2.1: the authors introduce the concept of cusp but do not really
define it

\bigskip

{\bf A:} We have included the following definition: ``The cusp is defined as the circle 
in the equatorial plane on which the pressure gradient vanishes and the angular 
momentum of the disk equals the Keplerian angular momentum." 

\bigskip

{\bf R:} - Sec. 2.1: they introduce the equation of state (13) but do not discuss
how they treat the specific internal energy that is part of the
enthalpy used in (13). A comment should be added.

\bigskip

{\bf A:} In our work we are following the same procedure as in Komissarov (2006), where no comment on the treatment
of the specific internal energy is explicitly provided. We have simply followed the same approach and have used the
same polytropic relationship between the enthalpy and the fluid pressure, Eq.~(12), and between the rest-mass density and the
fluid pressure (as can be read in the paragraph following Eq.~(27)). Thanks to the remark of the referee we now realise 
that by doing this we (and Komissarov) are making an implicit assumption about the internal energy. By looking at the 
expression of the stress-energy tensor in Eq.~(9) of the manuscript, we see that the fluid enthalpy $w$ includes the 
rest-mass density $\rho$, and can thus be defined as $w=\rho h$ where $h$ is the specific enthalpy. The relativistic definition 
of this quantity (in units with $c=1$) is $h=1+\varepsilon+p/\rho$, where $\varepsilon$ is the specific internal energy and 
$p$ is the fluid pressure. From a thermodynamical point of view a fluid is non-relativistic in the limit $h=1$. Therefore,
since both equations of state we use are polytropes (relating $p$ with $w$ or with $\rho$ in the same functional form) we 
conclude that we are implictly assuming that $h=1$ in our models, i.e.~the disks we build in our procedure are non-relativistic 
from a thermodynamical point of view.

\bigskip 

We have included the following comment at the end of the paragraph following Eq.~(27): 
``We notae that the fluid enthalpy $w$ includes the rest-mass density $\rho$, and can thus be defined as $w=\rho h$ where
$h$ is the specific enthalpy. The relativistic definition of this quantity is $h=1+\varepsilon+p/\rho$, where $\varepsilon$ is
the specific internal energy. From a thermodynamical point of view a non-relativistic fluid satisfies $h=1$. Therefore,
since we use poytropic equations of state (relating $p$ with either $w$ or $\rho$ in the same functional form) we are
implictly assuming that $h=1$, i.e.~the discs we build in our procedure are non-relativistic from a thermodynamical point of view."

\bigskip

To avoid confusion, we have changed the name of variable used for the step of the Simpson's rule from $h$ to $\Delta r$.

\bigskip

{\bf R:} - Sec. 2.1: I am not clear how the magnetic field is actually
prescribed. The author mention that the only nonzero component of
the magnetic field in the comoving frame is the azimuthal one but I am
still unclear how the strength of this component is specified. A
comment should be added.

\bigskip

{\bf A:} We have not included further details about the prescription of the magnetic field because, in practice, in our procedure we 
only need to use $p_m\equiv b^2/2$,
and not the individual components of the magnetic-field themselves. We believe that including those extra details in the manuscript does not
provide relevant information. If interested, the referee is addressed to appendix A of Komissarov 
(2006) for the derivation of the explicit relations (which we do not employ anyway).

\bigskip

{\bf R:} - Sec. 2.1: More on this topic: do the fluid isobaric surfaces (i.e. the
surfaces of constant fluid pressure) coincide with the magnetic
isobaric surfaces (i.e. the surfaces of constant magnetic pressure)?
How are these surfaces related to the iso-density surfaces? A comment
should be added.

\bigskip

{\bf A:} They do not coincide, as can be inferred from Eqs.~(12) and (14). Also notice that the values of $r_{\rm max}$ and $r_{\rm m,max}$ reported in the tables, do not coincide, which shows that fluid isobaric surfaces and magnetic isobaric surfaces cannot coincide. On the other hand, the iso-density surfaces and the fluid isobaric surfaces do coincide, since both variables are related by the barotropic equation of state.

\bigskip

We have included the following text at the end of the first paragraph of the results section: ``We note in particular that the radii of the maximum fluid pressure and magnetic pressure, $r_{{\rm{max}}}$ and $r_{{\mathrm{m, max}}}$, respectively, for the same value of $\beta_{\mathrm{m}_{\mathrm{c}}}$, never coincide. This also reflects the fact that constant fluid pressure surfaces do not coincide with constant magnetic pressure surfaces."

\bigskip

{\bf R:} - Sec. 3: the authors are a bit too sloppy when referring to the fluid
pressure, which they call respectively: ``fluid pressure", ``gas
pressure" ``thermal pressure", and sometimes simply as ``pressure". Since
distinguishing the fluid, magnetic and total pressure is important, I
ask the authors to always just talk of ``fluid pressure".

\bigskip

{\bf A:} We have followed the recomendation of the referee and in the revised version
of the paper only mentions to ``fluid pressure", ``magnetic pressure", and 
``total pressure" appear. Those appearances are highlighted in boldface
throughout.

\bigskip

{\bf R:} - Sec. 3: similar considerations apply to the density, which should
be referred to as ``rest-mass density"

\bigskip

{\bf A:} Done.

\bigskip

{\bf R:} - Sec. 3: the authors discuss that the rest-mass density distribution can
``be trivially obtained inverting the barotropic relation, $p = K
\rho^{\kappa}$". I am not sure this is so trivial unless an assumption is
made on the specific internal energy; the authors should discuss more
clearly how they actually obtain the density from Eqs. (12) and (14).

\bigskip

{\bf A:} 
We have addressed this issue previously, while responding to a related query raised by the referee.

%The specific internal energy $\varepsilon$ is obtained by combining the ideal fluid (gamma-law)
%equation of state which, in the notation used in the paper we write $P=(\kappa-1)\rho\varepsilon$,
%and the barotropic equation of state. Solving for $\varepsilon$ we obtain
%\begin{eqnarray*}
%\varepsilon=\frac{K}{\kappa-1}\rho^{\kappa-1}\,.
%\end{eqnarray*}
%We have incorporated the following statements at the end of the paragraph after Eq.~(27):
%``We note that the specific internal energy $\varepsilon$ is needed to compute the enthalpy. We obtain
%this quantity by combining the ideal fluid and the barotropic equations of state, which yields 
%$\varepsilon=\frac{K}{\kappa-1}\rho^{\kappa-1}$."


\bigskip

{\bf R:} - Sec. 3: the authors discuss that Eq.~(24) diverges for $\theta \rightarrow \pi/2$,
i.e. at the equator. I am a bit puzzled by this statement and I wonder if
this is an exact statement, ie whether it is true that

\begin{equation}
\partial_{\theta} W = 0 \ \mathrm{for} \ \theta \rightarrow \frac{\pi}{2}.
\end{equation}

If this is an exact result, I urge the authors to prove it. If it is
not, I urge the authors to simply mention that the gradients become
very small.

\bigskip

{\bf A:} We believe this is an exact result and present the proof below:

\bigskip

Eq.~(22) of the paper gives us

\begin{eqnarray*}
\partial_{\theta} W = \partial_{\theta} \ln|u_t| - \frac{\Omega \partial_{\theta}l}{1 - \Omega l}\, = 0.
\end{eqnarray*}

Then, we have to prove that either the two terms of this equation are equal or they  are identically zero at $\theta = \pi/2$. We shall see that the latter is true.
First, we take the derivative of the angular momentum distribution $\partial_{\theta} l$. From Eq.~(7) we can write the angular momentum distribution as $l(r, \theta) = l(r) \sin^{2\gamma} \theta$, so that the derivative reads
\begin{eqnarray*}
\partial_{\theta} l = 2\gamma l(r) \sin^{2\gamma - 1} \theta \cos \theta \,.
\end{eqnarray*}
Taking $\theta = \pi/2$ leads to $\partial_{\theta} l = 0$, so the second term equals to zero.
To show that the first term is also zero, we write it as
\begin{eqnarray*}
\partial_{\theta} \ln|u_t| = \partial_{\theta} \ln \left(\frac{\mathcal{L}}{\mathcal{A}}\right)^{\frac{1}{2}}\,,
\end{eqnarray*}
where $\mathcal{L} = g_{t \phi}^2 - g_{tt} g_{\phi\phi}$ and $\mathcal{A} = g_{\phi\phi} + 2 l g_{t\phi} + l^2g_{tt}$, and we have dropped the absolute value, as it is irrelevant to this discussion. Then, the derivative is
\begin{eqnarray*}
\partial_{\theta} \ln|u_t| = \frac{1}{2} \frac{\mathcal{A}}{\mathcal{L}}\frac{\partial_{\theta}{\mathcal{L}}\mathcal{A} - \mathcal{L}\partial_{\theta}\mathcal{A}}{\mathcal{A}^2} = \frac{1}{2} \frac{\partial_{\theta}{\mathcal{L}}\mathcal{A} - \mathcal{L}\partial_{\theta}\mathcal{A}}{\mathcal{A} \mathcal{L}}\,.
\end{eqnarray*}
We use Boyer-Lindquist coordinates, for which the metric components read
\begin{eqnarray*}
g_{tt} = - \left(1 - \frac{2Mr}{\rho^2}\right), \ g_{t\phi} = -\frac{2Mar\sin^2\theta}{\rho^2}, \ g_{\phi\phi} = \left(r^2 + a^2 + \frac{2Ma^2r\sin^2\theta}{\rho^2}\right) \sin^2 \theta\,,
\end{eqnarray*}
where $\rho^2 = r^2 + a^2\cos^2 \theta$. Therefore, we obtain $\mathcal{L} = \Delta \sin^2 \theta$, where $\Delta = r^2 - 2Mr + a^2$, and its derivative $\partial_{\theta}\mathcal{L} = 2 \Delta \sin \theta \cos \theta$, which is zero for $\theta = \pi/2$. Next, we take the derivative of $\mathcal{A}$
\begin{eqnarray*}
\partial_{\theta} \mathcal{A} = \partial_{\theta} g_{\phi\phi} + 2 (\partial_{\theta} l) g_{t\phi} + 2l\partial_{\theta} g_{t\phi} + 2l(\partial_{\theta} l) g_{tt} + l^2 \partial_{\theta} g_{tt} = \partial_{\theta} g_{\phi\phi} + 2l\partial_{\theta} g_{t\phi} + l^2 \partial_{\theta} g_{tt}\,,
\end{eqnarray*}
where we have used the previous result that $\partial_{\theta} l=0$ at $\theta = \pi/2$. By inspecting the metric components, it is easy to see that all terms depending on $\theta$ are functions of $\sin^2 \theta$ or $\cos^2 \theta$. This means their $\theta$ derivatives will have at least a $\cos \theta$ multiplying. Then $\partial_{\theta} \mathcal{A} = 0$ at $\theta = \pi/2$. Therefore, $\partial_{\theta} \ln|u_t|$ is also zero at $\theta = \pi/2$.

\bigskip

We have incorporated this simple demonstration as Appendix A in the manuscript.

\bigskip

We have also incorporated a second appendix in the manuscript, Appendix B, to explain why indeed, as shown in the inset of the right panel of Fig.~6, all models cross at $r_{\mathrm{m}_{\mathrm{max}}} = r_{\mathrm{c}}$ for a very specific value of $\beta_{\mathrm{m}_{\mathrm{c}}}$. Correspondingly, we have added the following text towards the end of Section 4: ``{\bf It must be noted that the location of the maximum of the magnetic pressure is identical for all models considered when $\beta_{\mathrm{m}_{\mathrm{c}}} \equiv 1/(\lambda - 1) = 3$, as we show in Appendix~B. At this value of $\beta_{\mathrm{m}_{\mathrm{c}}}$ all models cross at $r_{\mathrm{m}_{\mathrm{max}}} = r_{\mathrm{c}}$, as it is more clearly shown in the inset of the right panel of Fig.~6.}"

\bigskip

{\bf R:} - Fig. 1: as with the middle panel, I suggest that the right
panel should also show the logarithm of the rest-mass density.

\bigskip

{\bf A:} The reason we show the middle panel in logarithm scale and the right panel in linear scale is because this is the choice 
 in Komissarov (2006). We have followed the same choice in order to facilitate the qualitative and quantitative comparison
for this constant angular momentum test. If the referee concurred, we would prefer to keep the right panel in linear scale. If not, we would
not mind changing the right panel to logarithmic scale.

\bigskip

{\bf R:} - Fig. 2: all of the panels would benefit from a legend showing the
corresponding values of gamma and beta. This would avoid having to read the caption repeatedly. The same is true also for Fig. 3.

\bigskip

{\bf A:} Done.

\bigskip

{\bf R:} - Fig. 4: the caption should be corrected in ``$\beta = 10^3$ $\rightarrow $ $\beta_{m_c}
= 10^3$".

\bigskip

{\bf A:} Done.

\bigskip

{\bf R:} - Fig. 4: the authors do not discuss what is the largest magnetization
that they can handle. I agree that $\beta_{m_c} = 10^{-3}$ is probably
already quite high, but what happens for say $\beta_{m_c} = 10^{-6}$? Can
a solution still be found?

\bigskip

{\bf A:} We have built models with very small values of $\beta_{m_c}$, such as  $\beta_{m_c} = 10^{-20}$ (extremely high magnetisation) without encountering numerical difficulties. In practice we have found that there are no qualitative differences in the structure of the discs once $\beta_{m_c}$ becomes smaller than a sufficiently small value, about $\beta_{m_c}=10^{-3}$. This is
the reason why we choose that particular value as our lower limit in the figures and tables of the manuscript. 

\bigskip

{\bf R:} - Sec. 4: I find the discussion in this section the less clear; a lot
of the considerations appear phenomenological but I can't find the
simple back-of-the-envelope demonstration that what the authors observe
numerically is what should be expected on physical grounds. For
example, the authors find that the $\rho_{\mathrm{max}}$ increases with magnetization
(left panel of Fig. 6) or that the position of the rest-mass density
maximum moves in with increasing magnetization (middle panel of
Fig. 6). What are the physical reasons for this behavior? I have some
suggestions on this but I urge the authors to provide some simple and
intuitive explanations first.

\bigskip

{\bf A:}

\bigskip

{\bf R:} - References: the list is already quite complete but maybe it can be
complemented with the following ones, many of which are actually
written by one of the authors.

\bigskip

A comprehensive discussion of the properties of tori from merging binaries:

{\tt http://adsabs.harvard.edu/abs/2016arXiv160703540B}

\bigskip

First detailed discussion of non-constant specific angular momentum tori:

{\tt http://adsabs.harvard.edu/abs/2004MNRAS.354.1040M}

\bigskip

First discussion of the dynamics of magnetized tori built with the Komissarov solution

{\tt http://adsabs.harvard.edu/abs/2007MNRAS.378.1101M}

\bigskip

{\bf A:} We have incorporated in the revised version of the article the first and the third references 
suggested by the referee. However, we have not incorporated the second 
reference. While this work is relevant in the context of oscillations 
of non-constant-angular momentum disks, including this reference in our 
work seems to us a bit forced since it discusses aspects which are a bit 
away from the main topic under discussion in our paper.

%%%%%%%%%%%%%%%%%%%%%%%%%%%%%%%%%%%%%%%%%%%%%%%%%%%%%%%%%%%%%%%%%%%%%%%%%%%%%%%  
%\begin{small}
%\begin{thebibliography}{99}
%%%%%%%%%%%%%%%%%%%%%%%%%%%%%%%%%%%%%%%%%%%%%%%%%%%%%%%%%%%%%%%%%%%%%%%%%%%%%%%%  
%%
%
%%  
%\end{thebibliography}

% \end{small}

 \end{document}
